\documentclass[a4paper,11pt]{article}
\usepackage{geometry}
\usepackage{amsmath}
\usepackage{amssymb}
\usepackage{array}
\usepackage{tabularx}
\usepackage{setspace}
\usepackage{graphicx} % Required for including images
\usepackage{tikz}     % Required for watermarking
\usepackage{fancyhdr} % Required for custom headers/footers
\usepackage{xcolor}
\usepackage{tcolorbox}

\geometry{a4paper, margin=1in}
\setlength{\headheight}{33pt} % Fix header height for logo
\addtolength{\topmargin}{-21pt} % Adjust top margin accordingly

% --- Watermark setup ---
\usepackage{eso-pic}
\newcommand\BackgroundPicture{%
\put(\LenToUnit{0.5\paperwidth},\LenToUnit{0.5\paperheight}){%
\makebox(0,0)[c]{%
\tikz[opacity=0.08]\node{\includegraphics[width=1\paperwidth]{../oatutors_logo.png}};%
}%
}}
\AddToShipoutPicture{\BackgroundPicture}
% --- End Watermark setup ---

% --- Header setup for logo ---
\pagestyle{fancy}
\fancyhf{} % Clear all header and footer fields
\fancyhead[L]{\includegraphics[height=1.0cm]{../oatutors_logo.png}} % Logo on the left (slightly smaller for better proportion)
\fancyhead[C]{\textbf{11+ Mathematics - Student Notes}} % Course title in center
\fancyhead[R]{\rightmark} % Section title on the right
\fancyfoot[C]{\thepage} % Page number in the footer
\renewcommand{\headrulewidth}{0.4pt} % Line under the header
\renewcommand{\footrulewidth}{0.4pt} % Line over the footer
% --- End Header setup ---

\begin{document}
\onehalfspacing

% This resets the page counter to 0 for the first page after the header setup
% \thispagestyle{fancy} % Apply fancy style to the first page too

\begin{center}
\textbf{\Large 11+ Maths Student Notes: Place Value}
\vspace{0.2cm}
\end{center}

\hrule
\vspace{0.1cm}

\textbf{Website:} \texttt{https://oatutors.co.uk/}

\vspace{0.2cm}
\hrule
\vspace{0.3cm}

\begin{tcolorbox}[colback=blue!5!white,colframe=blue!75!black,title=\textbf{What You'll Learn Today}]
\begin{itemize}
    \item Read and write large numbers (up to millions!)
    \item Understand what each digit means in a number
    \item Compare and order numbers using $>$, $<$, $=$
    \item Work with decimal numbers confidently
\end{itemize}
\end{tcolorbox}

\section{The Place Value System}

\begin{center}
\begin{tabular}{|c|c|c|c|c|c|c|}
\hline
\textbf{Millions} & \textbf{HTh} & \textbf{TTh} & \textbf{Th} & \textbf{H} & \textbf{T} & \textbf{U} \\
\hline
M & HTh & TTh & Th & H & T & U \\
\hline
3 & 2 & 4 & 7 & 5 & 6 & 8 \\
\hline
\end{tabular}
\end{center}

\textbf{This number is:} 3,247,568
\\
\textbf{In words:} Three million, two hundred and forty-seven thousand, five hundred and sixty-eight

\begin{tcolorbox}[colback=green!5!white,colframe=green!75!black,title=\textbf{Key Rule}]
Each position is 10 times bigger than the position to its right!
\end{tcolorbox}

\section{Decimal Place Value}

\begin{center}
\begin{tabular}{|c|c|c|c|c|c|c|}
\hline
\textbf{H} & \textbf{T} & \textbf{U} & \textbf{.} & \textbf{Tenths} & \textbf{Hundredths} & \textbf{Thousandths} \\
\hline
4 & 3 & 7 & . & 2 & 6 & 8 \\
\hline
\end{tabular}
\end{center}

\textbf{This number is:} 437.268
\\
\textbf{In words:} Four hundred and thirty-seven point two six eight

\begin{tcolorbox}[colback=orange!5!white,colframe=orange!75!black,title=\textbf{Important!}]
\begin{itemize}
    \item We only say "and" for the decimal point
    \item Don't say "point two hundred and sixty-eight"
    \item Say each digit separately: "point two six eight"
\end{itemize}
\end{tcolorbox}

\section{Comparing Numbers}

\subsection{The Strategy}
\begin{enumerate}
    \item Start from the left (biggest place value)
    \item Compare digits in the same position
    \item The first different digit tells you which is bigger
\end{enumerate}

\subsection{Examples}
\begin{itemize}
    \item 34,567 $>$ 34,476 (Compare hundreds: 5 $>$ 4)
    \item 7.39 $<$ 7.4 (7.4 = 7.40, so 39 hundredths $<$ 40 hundredths)
\end{itemize}

\section{Rounding Numbers}

\begin{tcolorbox}[colback=red!5!white,colframe=red!75!black,title=\textbf{Rounding Rules}]
\textbf{Look at the digit to the RIGHT of where you're rounding:}
\begin{itemize}
    \item If it's 5 or more → round UP
    \item If it's 4 or less → round DOWN
\end{itemize}
\end{tcolorbox}

\subsection{Examples}
\begin{itemize}
    \item 47,368 rounded to nearest thousand = 47,000
    \item 3.247 rounded to 1 decimal place = 3.2
    \item 156,499 rounded to nearest hundred thousand = 200,000
\end{itemize}

\section{Quick Reference}

\begin{center}
\begin{tabular}{|l|l|}
\hline
\textbf{Symbol} & \textbf{Meaning} \\
\hline
$>$ & Greater than (bigger) \\
\hline
$<$ & Less than (smaller) \\
\hline
$=$ & Equal to (same as) \\
\hline
\end{tabular}
\end{center}

\section{Common Mistakes to Avoid}

\begin{tcolorbox}[colback=purple!5!white,colframe=purple!75!black,title=\textbf{Watch Out!}]
\begin{itemize}
    \item Writing 20,345 as 2,0345 ❌
    \item Forgetting zeros as placeholders
    \item Mixing up tenths and hundredths
    \item Not lining up decimal points when comparing
\end{itemize}
\end{tcolorbox}

\section{Memory Tricks}

\begin{itemize}
    \item \textbf{Thousands:} Think "thousand" has 4 letters, thousands have 4 digits (1,000)
    \item \textbf{Decimals:} Tenths are bigger than hundredths (like £0.10 $>$ £0.01)
    \item \textbf{Comparing:} The "mouth" of $>$ opens to the bigger number
\end{itemize}

\section{Quick Practice}

\textbf{Write in figures:}
\begin{enumerate}
    \item Two million, four hundred and fifty thousand, six hundred and twelve
    \item Seven point zero three five
\end{enumerate}

\textbf{Compare using $>$, $<$, or $=$:}
\begin{enumerate}
    \setcounter{enumi}{2}
    \item 45,670 \_\_\_ 45,607
    \item 0.5 \_\_\_ 0.50
\end{enumerate}

\textbf{Round to the nearest hundred:}
\begin{enumerate}
    \setcounter{enumi}{4}
    \item 4,567 = \_\_\_\_\_
    \item 23,449 = \_\_\_\_\_
\end{enumerate}

\vspace{1cm}

\begin{tcolorbox}[colback=gray!10!white,colframe=gray!50!black,title=\textbf{Homework Reminder}]
\begin{itemize}
    \item Practice reading large numbers aloud
    \item Complete the place value worksheet
    \item Use real money to practice decimals!
\end{itemize}
\end{tcolorbox}

\end{document}