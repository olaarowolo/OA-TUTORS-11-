\documentclass{article}
\usepackage{geometry}
\usepackage{amsmath}
\usepackage{amssymb}
\usepackage{array}
\usepackage{tabularx}
\usepackage{setspace}
\usepackage{graphicx}
\usepackage{tikz}
\usepackage{fancyhdr}

\geometry{a4paper, margin=1in}
\setlength{\headheight}{33pt}
\addtolength{\topmargin}{-21pt}

% --- Watermark setup ---
\usepackage{eso-pic}
\newcommand\BackgroundPicture{%
\put(\LenToUnit{0.5\paperwidth},\LenToUnit{0.5\paperheight}){%
\makebox(0,0)[c]{%
\tikz[opacity=0.08]\node{\includegraphics[width=1\paperwidth]{../oatutors_logo.png}};%
}%
}}
\AddToShipoutPicture{\BackgroundPicture}
% --- End Watermark setup ---

% --- Header setup for logo ---
\pagestyle{fancy}
\fancyhf{}
\fancyhead[L]{\includegraphics[height=1.0cm]{../oatutors_logo.png}}
\fancyhead[C]{\textbf{11+ Mathematics}}
\fancyhead[R]{\rightmark}
\fancyfoot[C]{\thepage}
\renewcommand{\headrulewidth}{0.4pt}
\renewcommand{\footrulewidth}{0.4pt}
% --- End Header setup ---

\begin{document}
\onehalfspacing

\begin{center}
\textbf{\Large 11+ Maths: Number Patterns \& Sequences}
\vspace{0.2cm}
\end{center}

\hrule
\vspace{0.1cm}

\textbf{Topic:} Linear and Non-linear Sequences \\
\textbf{Target Age:} 10-11 years (Preparing for 11+ Exams) \\
\textbf{Time:} 60 Minutes \\
\textbf{Resources:} Number sequences, pattern blocks, sequence cards \\
\textbf{Website:} \texttt{https://oatutors.co.uk/}

\vspace{0.2cm}
\hrule
\vspace{0.3cm}

This lesson develops pattern recognition skills essential for 11+ success. Students will explore arithmetic sequences, geometric patterns, and special number sequences commonly tested in entrance exams.

\section{Learning Objectives}
By the end of this lesson, students will be able to:
\begin{itemize}
    \item Identify and continue arithmetic sequences (linear patterns)
    \item Recognize geometric sequences and other non-linear patterns
    \item Find the nth term of simple arithmetic sequences
    \item Work with Fibonacci sequences and other special patterns
    \item Apply pattern recognition to solve 11+ problems
    \item Generate their own number sequences with rules
\end{itemize}

\section{Starter Activity (10 Minutes)}

\begin{tabularx}{\textwidth}{|>{\raggedright\arraybackslash}p{1cm}|>{\raggedright\arraybackslash}p{3cm}|>{\raggedright\arraybackslash}X|}
\hline
\textbf{Time} & \textbf{Activity} & \textbf{Description} \\
\hline
5 mins & Pattern Detective & Show sequences: 3, 6, 9, 12, \_\_; 2, 4, 8, 16, \_\_; 1, 1, 2, 3, 5, \_\_. Students identify patterns and next terms. \\
\hline
5 mins & Human Sequence & Students form arithmetic sequence: start with 7, add 4 each time. Class predicts who comes next. \\
\hline
\end{tabularx}

\section{Main Teaching (25 Minutes)}

\subsection{Linear Sequences (Arithmetic Sequences) (10 minutes)}

\textbf{Definition:} Sequences where the same number is added (or subtracted) each time.

\textbf{Key Examples:}
\begin{center}
\begin{tabular}{|c|c|c|c|}
\hline
\textbf{Sequence} & \textbf{First Term} & \textbf{Common Difference} & \textbf{Rule} \\
\hline
5, 8, 11, 14, 17... & 5 & +3 & Start 5, add 3 \\
\hline
20, 17, 14, 11, 8... & 20 & -3 & Start 20, subtract 3 \\
\hline
4, 9, 14, 19, 24... & 4 & +5 & Start 4, add 5 \\
\hline
\end{tabular}
\end{center}

\textbf{Finding the Pattern:}
1. Look at differences between consecutive terms
2. If differences are constant = arithmetic sequence
3. Formula: Term = First term + (Position - 1) × Common difference

\textbf{Example:} 7, 11, 15, 19, 23...
\begin{itemize}
    \item Differences: +4, +4, +4, +4 (constant)
    \item 10th term = 7 + (10-1) × 4 = 7 + 36 = 43
\end{itemize}

\subsection{Non-Linear Sequences (8 minutes)}

\textbf{Quadratic Sequences:} Based on square numbers
\begin{itemize}
    \item 1, 4, 9, 16, 25... (perfect squares: n²)
    \item 2, 8, 18, 32, 50... (2n²)
    \item 0, 3, 8, 15, 24... (n² - 1)
\end{itemize}

\textbf{Geometric Sequences:} Each term is multiplied by same number
\begin{itemize}
    \item 2, 6, 18, 54, 162... (×3 each time)
    \item 5, 10, 20, 40, 80... (×2 each time)  
    \item 81, 27, 9, 3, 1... (÷3 each time)
\end{itemize}

\textbf{Special Sequences:}
\begin{itemize}
    \item \textbf{Fibonacci:} 1, 1, 2, 3, 5, 8, 13... (add previous two terms)
    \item \textbf{Triangular:} 1, 3, 6, 10, 15, 21... (T$_n$ = $\frac{n(n+1)}{2}$)
    \item \textbf{Prime numbers:} 2, 3, 5, 7, 11, 13, 17...
\end{itemize}

\subsection{Pattern Recognition Strategies (7 minutes)}

\textbf{11+ Method for Any Sequence:}
\begin{enumerate}
    \item Find first differences (subtract consecutive terms)
    \item If constant → arithmetic sequence  
    \item If not constant, find second differences
    \item If second differences constant → quadratic sequence
    \item Look for multiplication/division patterns
    \item Check for special sequences (squares, cubes, primes)
\end{enumerate}

\textbf{Example Analysis:} 3, 7, 13, 21, 31...
\begin{center}
\begin{tabular}{|c|c|c|c|c|}
\hline
Terms: & 3 & 7 & 13 & 21 & 31 \\
\hline
1st differences: & & 4 & 6 & 8 & 10 \\
\hline  
2nd differences: & & & 2 & 2 & 2 \\
\hline
\end{tabular}
\end{center}
Second differences are constant (2), so this is quadratic: n² + n + 1

\section{Guided Practice (15 Minutes)}

\subsection*{Worksheet: Sequence Mastery}

\textbf{Section A: Continue the Sequences (5 minutes)}
\begin{enumerate}
    \item 4, 7, 10, 13, 16, \_\_\_, \_\_\_
    \item 2, 6, 18, 54, \_\_\_, \_\_\_  
    \item 1, 4, 9, 16, 25, \_\_\_, \_\_\_
    \item 50, 47, 44, 41, \_\_\_, \_\_\_
    \item 1, 1, 2, 3, 5, 8, \_\_\_, \_\_\_
\end{enumerate}

\textbf{Section B: Find the Rule (5 minutes)}
\begin{enumerate}
    \setcounter{enumi}{5}
    \item What's the rule for: 5, 9, 13, 17, 21...?
    \item What's the 20th term in the sequence: 3, 8, 13, 18, 23...?
    \item Which sequence has the rule "multiply previous term by 2"?
    \item Describe the pattern: 2, 5, 10, 17, 26...
\end{enumerate}

\textbf{Section C: Create Sequences (5 minutes)}
\begin{enumerate}
    \setcounter{enumi}{9}
    \item Write an arithmetic sequence starting with 12, common difference -5
    \item Create a sequence where each term is double the position number plus 1
    \item Write the first 6 terms of the sequence n² + 3
\end{enumerate}

\section{Independent Work (8 Minutes)}

\textbf{11+ Challenge Problems:}
\begin{enumerate}
    \item The sequence 5, 8, 13, 20, 29... follows the rule n² + 4. What is the 10th term?
    \item A sequence starts 3, 6, 12, 24... If this pattern continues, what is the 8th term?
    \item Find the missing term: 2, 7, 14, 23, \_\_\_, 47
    \item The nth term of a sequence is 4n - 1. Which term has value 39?
    \item In the Fibonacci sequence, what is the sum of the 5th and 6th terms?
\end{enumerate}

\section{Plenary and Assessment (2 Minutes)}

\begin{tabularx}{\textwidth}{|>{\raggedright\arraybackslash}p{1cm}|>{\raggedright\arraybackslash}p{3cm}|>{\raggedright\arraybackslash}X|}
\hline
\textbf{Time} & \textbf{Activity} & \textbf{Description} \\
\hline
2 mins & Sequence Showcase & Students share one sequence they created. Class identifies the pattern type. Quick assessment of understanding. \\
\hline
\end{tabularx}

\section{Homework Assignment}

\textbf{Pattern Explorer Tasks:}
\begin{enumerate}
    \item Complete 15 sequence continuation problems
    \item Find nth terms for 5 arithmetic sequences  
    \item Investigate Pascal's triangle patterns
    \item Create a poster showing 3 different sequence types
    \item Extension: Research the golden ratio and Fibonacci connection
\end{enumerate}

\section{Extension Activities}

For more able students:
\begin{itemize}
    \item Explore sequences in nature (sunflower spirals, shell patterns)
    \item Investigate convergent and divergent sequences
    \item Create sequences using algebraic expressions
    \item Study the Tower of Hanoi problem and its sequence
\end{itemize}

\newpage

\section*{Answer Key - For Teachers}

\subsection*{Guided Practice Answers}

\textbf{Section A: Continue the Sequences}
\begin{enumerate}
    \item 19, 22 (arithmetic: +3)
    \item 162, 486 (geometric: ×3)
    \item 36, 49 (square numbers: 6², 7²)
    \item 38, 35 (arithmetic: -3)
    \item 13, 21 (Fibonacci: add previous two)
\end{enumerate}

\textbf{Section B: Find the Rule}  
\begin{enumerate}
    \setcounter{enumi}{5}
    \item Start with 5, add 4 each time (or 4n + 1)
    \item 20th term = 3 + (20-1) × 5 = 98
    \item 1, 2, 4, 8, 16, 32... (powers of 2)
    \item n² + 1 (differences of 3, 5, 7, 9... which differ by 2)
\end{enumerate}

\textbf{Section C: Create Sequences}
\begin{enumerate}
    \setcounter{enumi}{9}
    \item 12, 7, 2, -3, -8, -13...
    \item 3, 5, 7, 9, 11, 13... (2n + 1)
    \item 4, 7, 12, 19, 28, 39... (n² + 3)
\end{enumerate}

\subsection*{Independent Work Answers}
\begin{enumerate}
    \item 10² + 4 = 104
    \item 3 × 2⁷ = 3 × 128 = 384  
    \item 34 (pattern is n² + n: 2², 2²+2, 3²+3, 4²+4, 5²+5, 6²+6)
    \item 4n - 1 = 39, so 4n = 40, n = 10 (10th term)
    \item 5th term = 5, 6th term = 8, sum = 13
\end{enumerate}

\subsection*{Teaching Notes}
\begin{itemize}
    \item Use visual patterns (dot arrangements) for kinesthetic learners
    \item Common error: Students confuse position number with term value
    \item Emphasize systematic approach: always check differences first
    \item Link to real-world patterns: population growth, compound interest
    \item Practice speed of recognition - 11+ timing is crucial
    \item Use colored counters or blocks to build physical sequences
\end{itemize}

\subsection*{Differentiation Strategies}
\begin{itemize}
    \item \textbf{Support:} Provide difference tables ready-made
    \item \textbf{Challenge:} Introduce cubic sequences and recursive formulas  
    \item \textbf{Visual:} Use graph paper to plot sequence values
    \item \textbf{Practical:} Create sequences with manipulatives
\end{itemize}

\end{document}