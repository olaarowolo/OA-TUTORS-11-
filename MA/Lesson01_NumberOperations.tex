\documentclass{article}
\usepackage{geometry}
\usepackage{amsmath}
\usepackage{amssymb}
\usepackage{array}
\usepackage{tabularx}
\usepackage{setspace}
\usepackage{graphicx}
\usepackage{tikz}
\usepackage{fancyhdr}

\geometry{a4paper, margin=1in}
\setlength{\headheight}{33pt}
\addtolength{\topmargin}{-21pt}

% --- Watermark setup ---
\usepackage{eso-pic}
\newcommand\BackgroundPicture{%
\put(\LenToUnit{0.5\paperwidth},\LenToUnit{0.5\paperheight}){%
\makebox(0,0)[c]{%
\tikz[opacity=0.08]\node{\includegraphics[width=1\paperwidth]{../oatutors_logo.png}};%
}%
}}
\AddToShipoutPicture{\BackgroundPicture}
% --- End Watermark setup ---

% --- Header setup for logo ---
\pagestyle{fancy}
\fancyhf{}
\fancyhead[L]{\includegraphics[height=1.0cm]{../oatutors_logo.png}}
\fancyhead[C]{\textbf{11+ Mathematics}}
\fancyhead[R]{\rightmark}
\fancyfoot[C]{\thepage}
\renewcommand{\headrulewidth}{0.4pt}
\renewcommand{\footrulewidth}{0.4pt}
% --- End Header setup ---

\begin{document}
\onehalfspacing

\begin{center}
\textbf{\Large 11+ Maths: Number Operations}
\vspace{0.2cm}
\end{center}

\hrule
\vspace{0.1cm}

\textbf{Topic:} Addition, Subtraction, Multiplication, Division \\
\textbf{Target Age:} 10-11 years (Preparing for 11+ Exams) \\
\textbf{Time:} 60 Minutes \\
\textbf{Resources:} Whiteboard, Worksheets, Calculator (for checking) \\
\textbf{Website:} \texttt{https://oatutors.co.uk/}

\vspace{0.2cm}
\hrule
\vspace{0.3cm}

This lesson focuses on mastering the four fundamental number operations essential for 11+ success. Students will practice mental arithmetic, written methods, and word problems using UK curriculum standards.

\section{Learning Objectives}
By the end of this lesson, students will be able to:
\begin{itemize}
    \item Apply efficient mental strategies for addition and subtraction
    \item Use standard written methods for multiplication and division
    \item Solve multi-step word problems involving all four operations
    \item Check answers using inverse operations
    \item Work confidently with numbers up to 10,000
\end{itemize}

\section{Starter Activity (10 Minutes)}

\begin{tabularx}{\textwidth}{|>{\raggedright\arraybackslash}p{1cm}|>{\raggedright\arraybackslash}p{3cm}|>{\raggedright\arraybackslash}X|}
\hline
\textbf{Time} & \textbf{Activity} & \textbf{Description} \\
\hline
5 mins & Mental Maths Warm-up & Quick-fire questions: 47 + 28, 83 - 39, 7 × 8, 72 ÷ 9. Use fingers, whiteboards for responses. \\
\hline
5 mins & Number Bond Revision & Review number bonds to 10, 20, and 100. Essential for mental calculation strategies. \\
\hline
\end{tabularx}

\section{Main Teaching (25 Minutes)}

\subsection{Addition and Subtraction Strategies (12 minutes)}

\textbf{Key Methods for 11+ Success:}

\begin{enumerate}
    \item \textbf{Column Method:} Essential for larger numbers
    \begin{center}
    \begin{tabular}{r}
        3847 \\
      + 2695 \\
        \hline
        6542
    \end{tabular}
    \end{center}
    
    \item \textbf{Mental Strategies:}
    \begin{itemize}
        \item Partitioning: 47 + 28 = (40 + 20) + (7 + 8) = 60 + 15 = 75
        \item Compensation: 47 + 29 = 47 + 30 - 1 = 77 - 1 = 76
        \item Near doubles: 47 + 46 = 46 + 46 + 1 = 92 + 1 = 93
    \end{itemize}
\end{enumerate}

\subsection{Multiplication and Division Methods (13 minutes)}

\textbf{Multiplication Strategies:}
\begin{enumerate}
    \item \textbf{Grid Method:} 
    \begin{center}
    \begin{tabular}{|c|c|c|}
    \hline
    × & 30 & 4 \\
    \hline
    20 & 600 & 80 \\
    \hline
    7 & 210 & 28 \\
    \hline
    \end{tabular}
    \end{center}
    34 × 27 = 600 + 80 + 210 + 28 = 918
    
    \item \textbf{Long Multiplication:} For 11+ standard
    
    \item \textbf{Mental Tricks:}
    \begin{itemize}
        \item Times by 10, 100, 1000: Move digits left
        \item Times by 5: Half and times by 10
        \item Times by 9: Times by 10, subtract original
    \end{itemize}
\end{enumerate}

\textbf{Division Strategies:}
\begin{enumerate}
    \item \textbf{Short Division:} For single-digit divisors
    \item \textbf{Long Division:} For two-digit divisors (essential for 11+)
    \item \textbf{Mental Division:} Using times tables knowledge in reverse
\end{enumerate}

\section{Guided Practice (15 Minutes)}

\subsection*{Worksheet: Number Operations Practice}

\textbf{Section A: Mental Arithmetic (5 minutes)}
\begin{enumerate}
    \item 56 + 47 = \_\_\_
    \item 124 - 67 = \_\_\_
    \item 8 × 37 = \_\_\_
    \item 156 ÷ 12 = \_\_\_
    \item 2.5 × 8 = \_\_\_
\end{enumerate}

\textbf{Section B: Written Methods (7 minutes)}
\begin{enumerate}
    \setcounter{enumi}{5}
    \item Use column addition: 2847 + 1395 + 726 = \_\_\_
    \item Use grid method: 47 × 38 = \_\_\_
    \item Use long division: 1248 ÷ 24 = \_\_\_
\end{enumerate}

\textbf{Section C: Word Problems (3 minutes)}
\begin{enumerate}
    \setcounter{enumi}{8}
    \item A school orders 24 boxes of pencils. Each box contains 36 pencils. How many pencils in total?
    \item Sarah has £4.50. She buys 3 items costing £1.25 each. How much change does she get?
\end{enumerate}

\section{Independent Work (8 Minutes)}

Students complete challenging 11+ style questions:
\begin{enumerate}
    \item What is the missing number? 347 + \_\_\_ = 1000
    \item A factory produces 1,440 widgets per day. How many widgets are produced in a week?
    \item Find two numbers that multiply to give 144 and add to give 24.
    \item 7,245 ÷ 15 = \_\_\_ remainder \_\_\_
\end{enumerate}

\section{Plenary and Assessment (2 Minutes)}

\begin{tabularx}{\textwidth}{|>{\raggedright\arraybackslash}p{1cm}|>{\raggedright\arraybackslash}p{3cm}|>{\raggedright\arraybackslash}X|}
\hline
\textbf{Time} & \textbf{Activity} & \textbf{Description} \\
\hline
2 mins & Quick Assessment & Exit ticket: One addition, one multiplication, one word problem. Check understanding before next lesson. \\
\hline
\end{tabularx}

\section{Homework Assignment}

\textbf{Practice Sheet: Number Operations Mastery}
\begin{enumerate}
    \item Mental arithmetic: 20 mixed questions
    \item Written methods: 5 complex calculations
    \item Word problems: 3 multi-step scenarios
    \item Challenge: Find all factor pairs of 72
\end{enumerate}

\section{Extension Activities}

For more able students:
\begin{itemize}
    \item Investigate different multiplication methods (Russian peasant method)
    \item Create word problems for given calculations
    \item Explore patterns in division remainders
\end{itemize}

\newpage

\section*{Answer Key - For Teachers}

\subsection*{Guided Practice Answers}

\textbf{Section A: Mental Arithmetic}
\begin{enumerate}
    \item 56 + 47 = 103
    \item 124 - 67 = 57
    \item 8 × 37 = 296
    \item 156 ÷ 12 = 13
    \item 2.5 × 8 = 20
\end{enumerate}

\textbf{Section B: Written Methods}
\begin{enumerate}
    \setcounter{enumi}{5}
    \item 2847 + 1395 + 726 = 4968
    \item 47 × 38 = 1786
    \item 1248 ÷ 24 = 52
\end{enumerate}

\textbf{Section C: Word Problems}
\begin{enumerate}
    \setcounter{enumi}{8}
    \item 24 × 36 = 864 pencils
    \item £4.50 - (3 × £1.25) = £4.50 - £3.75 = £0.75
\end{enumerate}

\subsection*{Independent Work Answers}
\begin{enumerate}
    \item 347 + 653 = 1000
    \item 1,440 × 7 = 10,080 widgets
    \item 12 and 12 (12 × 12 = 144, 12 + 12 = 24)
    \item 7,245 ÷ 15 = 483 remainder 0
\end{enumerate}

\subsection*{Teaching Notes}
\begin{itemize}
    \item Emphasize checking answers using inverse operations
    \item Common errors: Place value mistakes in addition/subtraction
    \item Ensure students show working clearly - essential for 11+ marking
    \item Use real-world contexts to make problems engaging
    \item Differentiate by providing scaffolding for less confident students
\end{itemize}

\end{document}