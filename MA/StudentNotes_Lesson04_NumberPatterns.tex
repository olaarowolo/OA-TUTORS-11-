\documentclass[a4paper,11pt]{article}
\usepackage{geometry}
\usepackage{amsmath}
\usepackage{amssymb}
\usepackage{array}
\usepackage{tabularx}
\usepackage{setspace}
\usepackage{graphicx} % Required for including images
\usepackage{tikz}     % Required for watermarking
\usepackage{fancyhdr} % Required for custom headers/footers
\usepackage{xcolor}
\usepackage{tcolorbox}

\geometry{a4paper, margin=1in}
\setlength{\headheight}{33pt} % Fix header height for logo
\addtolength{\topmargin}{-21pt} % Adjust top margin accordingly

% --- Watermark setup ---
\usepackage{eso-pic}
\newcommand\BackgroundPicture{%
\put(\LenToUnit{0.5\paperwidth},\LenToUnit{0.5\paperheight}){%
\makebox(0,0)[c]{%
\tikz[opacity=0.08]\node{\includegraphics[width=1\paperwidth]{../oatutors_logo.png}};%
}%
}}
\AddToShipoutPicture{\BackgroundPicture}
% --- End Watermark setup ---

% --- Header setup for logo ---
\pagestyle{fancy}
\fancyhf{} % Clear all header and footer fields
\fancyhead[L]{\includegraphics[height=1.0cm]{../oatutors_logo.png}} % Logo on the left (slightly smaller for better proportion)
\fancyhead[C]{\textbf{11+ Mathematics - Student Notes}} % Course title in center
\fancyhead[R]{\rightmark} % Section title on the right
\fancyfoot[C]{\thepage} % Page number in the footer
\renewcommand{\headrulewidth}{0.4pt} % Line under the header
\renewcommand{\footrulewidth}{0.4pt} % Line over the footer
% --- End Header setup ---

\begin{document}
\onehalfspacing

% This resets the page counter to 0 for the first page after the header setup
% \thispagestyle{fancy} % Apply fancy style to the first page too

\begin{center}
\textbf{\Large 11+ Maths Student Notes: Number Patterns \& Sequences}
\vspace{0.2cm}
\end{center}

\hrule
\vspace{0.1cm}

\textbf{Website:} \texttt{https://oatutors.co.uk/}

\vspace{0.2cm}
\hrule
\vspace{0.3cm}

\begin{tcolorbox}[colback=blue!5!white,colframe=blue!75!black,title=\textbf{What You'll Learn Today}]
\begin{itemize}
    \item Spot patterns in number sequences
    \item Continue arithmetic (linear) sequences
    \item Recognize special sequences like Fibonacci
    \item Find formulas for the nth term
\end{itemize}
\end{tcolorbox}

\section{Arithmetic Sequences (Linear Patterns)}

\begin{tcolorbox}[colback=green!5!white,colframe=green!75!black,title=\textbf{What's an Arithmetic Sequence?}]
A sequence where you ADD (or subtract) the same number each time
\end{tcolorbox}

\subsection{Examples}
\begin{itemize}
    \item 5, 8, 11, 14, 17... (add 3 each time)
    \item 20, 17, 14, 11, 8... (subtract 3 each time)  
    \item 4, 9, 14, 19, 24... (add 5 each time)
\end{itemize}

\subsection{How to Spot the Pattern}
\begin{enumerate}
    \item Find the differences between consecutive terms
    \item If the differences are constant → arithmetic sequence!
    \item The constant difference is called the "common difference"
\end{enumerate}

\textbf{Example:} 7, 11, 15, 19, 23...
\\Differences: +4, +4, +4, +4 ✓ (arithmetic sequence)

\section{Finding the nth Term}

\begin{tcolorbox}[colback=orange!5!white,colframe=orange!75!black,title=\textbf{The Formula}]
\textbf{nth term = First term + (Position - 1) × Common difference}
\end{tcolorbox}

\textbf{Example:} 7, 11, 15, 19, 23...
\begin{itemize}
    \item First term = 7
    \item Common difference = +4  
    \item 10th term = 7 + (10-1) × 4 = 7 + 36 = 43
\end{itemize}

\section{Non-Linear Sequences}

\subsection{Quadratic Sequences (Square Number Based)}
\begin{itemize}
    \item 1, 4, 9, 16, 25... (perfect squares: n²)
    \item 2, 8, 18, 32, 50... (2n²)
    \item 0, 3, 8, 15, 24... (n² - 1)
\end{itemize}

\subsection{Geometric Sequences (Multiplication/Division)}
\begin{itemize}
    \item 2, 6, 18, 54, 162... (×3 each time)
    \item 5, 10, 20, 40, 80... (×2 each time)
    \item 81, 27, 9, 3, 1... (÷3 each time)
\end{itemize}

\section{Special Sequences You Should Know}

\begin{tcolorbox}[colback=red!5!white,colframe=red!75!black,title=\textbf{Famous Sequences}]
\textbf{Fibonacci:} 1, 1, 2, 3, 5, 8, 13, 21... (add the previous two numbers)

\textbf{Triangular:} 1, 3, 6, 10, 15, 21... (T$_n$ = $\frac{n(n+1)}{2}$)

\textbf{Square:} 1, 4, 9, 16, 25, 36... (n²)

\textbf{Cube:} 1, 8, 27, 64, 125... (n³)

\textbf{Prime:} 2, 3, 5, 7, 11, 13, 17...
\end{tcolorbox}

\section{Pattern Detective Strategy}

\begin{tcolorbox}[colback=purple!5!white,colframe=purple!75!black,title=\textbf{How to Solve Any Sequence}]
\textbf{Step 1:} Find first differences (subtract consecutive terms)

\textbf{Step 2:} If constant → arithmetic sequence ✓

\textbf{Step 3:} If not constant, find second differences  

\textbf{Step 4:} If second differences constant → quadratic ✓

\textbf{Step 5:} Look for ×/÷ patterns

\textbf{Step 6:} Check for special sequences (squares, Fibonacci, etc.)
\end{tcolorbox}

\section{Worked Example}

\textbf{Find the pattern:} 3, 7, 13, 21, 31...

\begin{center}
\begin{tabular}{|c|c|c|c|c|}
\hline
Terms: & 3 & 7 & 13 & 21 & 31 \\
\hline
1st differences: & & 4 & 6 & 8 & 10 \\
\hline
2nd differences: & & & 2 & 2 & 2 \\
\hline
\end{tabular}
\end{center}

Second differences are constant (2) → This is a quadratic sequence!
\\The formula is: n² + n + 1

\section{Quick Practice}

\textbf{Continue these sequences:}
\begin{enumerate}
    \item 4, 7, 10, 13, \_\_\_, \_\_\_
    \item 2, 6, 18, 54, \_\_\_, \_\_\_
    \item 1, 4, 9, 16, \_\_\_, \_\_\_
    \item 1, 1, 2, 3, 5, \_\_\_, \_\_\_
\end{enumerate}

\textbf{Find the rules:}
\begin{enumerate}
    \setcounter{enumi}{4}
    \item What's the 20th term of: 3, 8, 13, 18, 23...?
    \item What type of sequence is: 5, 10, 20, 40, 80...?
\end{enumerate}

\section{Memory Tricks}

\begin{itemize}
    \item \textbf{Arithmetic:} Think "add the same amount" 
    \item \textbf{Geometric:} Think "multiply by the same amount"
    \item \textbf{Quadratic:} Look for second differences = constant
    \item \textbf{Fibonacci:} Each term = sum of previous two
\end{itemize}

\vspace{1cm}

\begin{tcolorbox}[colback=gray!10!white,colframe=gray!50!black,title=\textbf{Top Tips for 11+ Success}]
\begin{itemize}
    \item Always check differences systematically
    \item Learn the first 10 terms of common sequences
    \item Practice finding nth terms quickly
    \item Look for patterns in real life (spiral shells, flower petals!)
\end{itemize}
\end{tcolorbox}

\end{document}