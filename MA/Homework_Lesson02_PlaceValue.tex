\documentclass{article}
\usepackage{geometry}
\usepackage{amsmath}
\usepackage{amssymb}
\usepackage{array}
\usepackage{tabularx}
\usepackage{setspace}
\usepackage{graphicx} % Required for including images
\usepackage{tikz}     % Required for watermarking
\usepackage{fancyhdr} % Required for custom headers/footers
\usepackage{xcolor}

\geometry{a4paper, margin=1in}
\setlength{\headheight}{33pt} % Fix header height for logo
\addtolength{\topmargin}{-21pt} % Adjust top margin accordingly

% --- Watermark setup ---
\usepackage{eso-pic}
\newcommand\BackgroundPicture{%
\put(\LenToUnit{0.5\paperwidth},\LenToUnit{0.5\paperheight}){%
\makebox(0,0)[c]{%
\tikz[opacity=0.08]\node{\includegraphics[width=1\paperwidth]{../oatutors_logo.png}};%
}%
}}
\AddToShipoutPicture{\BackgroundPicture}
% --- End Watermark setup ---

% --- Header setup for logo ---
\pagestyle{fancy}
\fancyhf{} % Clear all header and footer fields
\fancyhead[L]{\includegraphics[height=1.0cm]{../oatutors_logo.png}} % Logo on the left (slightly smaller for better proportion)
\fancyhead[C]{\textbf{11+ Mathematics - Homework Exercise}} % Course title in center
\fancyhead[R]{\rightmark} % Section title on the right
\fancyfoot[C]{\thepage} % Page number in the footer
\renewcommand{\headrulewidth}{0.4pt} % Line under the header
\renewcommand{\footrulewidth}{0.4pt} % Line over the footer
% --- End Header setup ---

\begin{document}
\onehalfspacing

% This resets the page counter to 0 for the first page after the header setup
% \thispagestyle{fancy} % Apply fancy style to the first page too

\begin{center}
\textbf{\Large Homework Exercise: Place Value Mastery}
\vspace{0.2cm}
\end{center}

\hrule
\vspace{0.1cm}

\textbf{Name:} \underline{\hspace{4cm}} \quad \textbf{Date:} \underline{\hspace{3cm}} \quad \textbf{Class:} \underline{\hspace{2cm}} \\
\textbf{Lesson Topic:} Read, Write, Compare, and Order Numbers \\
\textbf{Website:} \texttt{https://oatutors.co.uk/}

\vspace{0.2cm}
\hrule
\vspace{0.3cm}

\section*{Instructions}
Complete all sections carefully. Pay attention to place value positions. Remember to line up decimal points when comparing.

\section{Section A: Converting Between Words and Figures (10 Questions)}

\subsection*{Part 1: Write in figures}
\begin{enumerate}
    \item Two million, three hundred and forty-five thousand, six hundred and twelve
    
    Answer: \underline{\hspace{4cm}}
    
    \item One million, four hundred and seven thousand, three hundred and eighty
    
    Answer: \underline{\hspace{4cm}}
    
    \item Five million, sixty thousand and nine
    
    Answer: \underline{\hspace{4cm}}
    
    \item Three hundred and twenty-seven thousand, four hundred and fifty-six
    
    Answer: \underline{\hspace{4cm}}
    
    \item Nine million, eight hundred thousand and seventy-three
    
    Answer: \underline{\hspace{4cm}}
\end{enumerate}

\subsection*{Part 2: Write in words}
\begin{enumerate}
    \setcounter{enumi}{5}
    \item 1,407,380
    
    \vspace{1cm}
    
    \item 3,045,672
    
    \vspace{1cm}
    
    \item 6,200,509
    
    \vspace{1cm}
    
    \item 847,036
    
    \vspace{1cm}
    
    \item 2,008,401
    
    \vspace{1cm}
\end{enumerate}

\section{Section B: Place Value Understanding (15 Questions)}

\subsection*{Question 1}
In the number 5,847,329, what is the value of:
\begin{enumerate}
    \item The digit 8: \underline{\hspace{3cm}}
    \item The digit 4: \underline{\hspace{3cm}}
    \item The digit 2: \underline{\hspace{3cm}}
    \item The digit 5: \underline{\hspace{3cm}}
\end{enumerate}

\subsection*{Question 2}
Write these decimals in the correct place value chart:

\begin{center}
\begin{tabular}{|c|c|c|c|c|c|c|}
\hline
\textbf{Hundreds} & \textbf{Tens} & \textbf{Units} & \textbf{.} & \textbf{Tenths} & \textbf{Hundredths} & \textbf{Thousandths} \\
\hline
 & & & . & & & \\
\hline
 & & & . & & & \\
\hline
 & & & . & & & \\
\hline
\end{tabular}
\end{center}

Numbers to place: 47.385, 156.07, 9.246

\subsection*{Question 3}
Fill in the missing numbers:
\begin{enumerate}
    \item 4,000 + 300 + \underline{\hspace{1cm}} + 6 = 4,376
    \item \underline{\hspace{1cm}} + 80,000 + 400 + 7 = 287,407
    \item 50,000 + \underline{\hspace{1cm}} + 90 + 3 = 53,093
    \item 6,000,000 + \underline{\hspace{1cm}} + 800 = 6,045,800
\end{enumerate}

\section{Section C: Comparing and Ordering Numbers (10 Questions)}

\subsection*{Question 1: Fill in $>$, $<$, or $=$}
\begin{enumerate}
    \item 67,234 \underline{\hspace{1cm}} 67,324
    \item 145,680 \underline{\hspace{1cm}} 145,608  
    \item 0.7 \underline{\hspace{1cm}} 0.67
    \item 3.45 \underline{\hspace{1cm}} 3.450
    \item 89,999 \underline{\hspace{1cm}} 90,001
\end{enumerate}

\subsection*{Question 2: Order from smallest to largest}
\begin{enumerate}
    \item 45,670 | 45,607 | 45,760 | 45,067
    
    \vspace{1cm}
    
    \item 0.45 | 0.405 | 0.54 | 0.5
    
    \vspace{1cm}
    
    \item 3.2 | 3.02 | 3.202 | 3.22
    
    \vspace{1cm}
    
    \item 156,780 | 156,087 | 156,708 | 156,807
    
    \vspace{1cm}
    
    \item 0.306 | 0.36 | 0.063 | 0.6
    
    \vspace{1cm}
\end{enumerate}

\section{Section D: Rounding Numbers (10 Questions)}

\begin{enumerate}
    \item Round 47,368 to the nearest thousand: \underline{\hspace{3cm}}
    \item Round 3.247 to 1 decimal place: \underline{\hspace{3cm}}
    \item Round 156,499 to the nearest hundred thousand: \underline{\hspace{3cm}}
    \item Round 678.956 to the nearest whole number: \underline{\hspace{3cm}}
    \item Round 34,567 to the nearest hundred: \underline{\hspace{3cm}}
    \item Round 0.8739 to 2 decimal places: \underline{\hspace{3cm}}
    \item Round 987,654 to the nearest ten thousand: \underline{\hspace{3cm}}
    \item Round 45.67 to the nearest ten: \underline{\hspace{3cm}}
    \item Round 2.3967 to 3 decimal places: \underline{\hspace{3cm}}
    \item Round 567,832 to the nearest million: \underline{\hspace{3cm}}
\end{enumerate}

\section{Section E: Word Problems (5 Questions)}

\subsection*{Question 1}
In the number 5,\_4\_,273, what digits could go in the spaces to make the smallest possible number?

\vspace{2cm}

\subsection*{Question 2}
A number rounded to the nearest thousand is 47,000. What is the smallest whole number it could be? What is the largest whole number it could be?

\vspace{2cm}

\subsection*{Question 3}
The population of a town is 67,845. Round this to the nearest hundred and to the nearest thousand. What is the difference between these two rounded numbers?

\vspace{2cm}

\subsection*{Question 4}
Arrange these numbers in descending order: $\frac{3}{4}$, 0.8, 0.73, $\frac{4}{5}$

\vspace{2cm}

\subsection*{Question 5}
The digit 7 appears in a 6-digit number. Its value is 70,000. In which position is the digit 7?

\vspace{2cm}

\section{Extension Activity: Number Investigation}
Create three different 6-digit numbers using the digits 2, 4, 6, 7, 8, 9 (use each digit exactly once in each number):

\begin{enumerate}
    \item The largest possible number: \underline{\hspace{4cm}}
    \item The smallest possible number: \underline{\hspace{4cm}}
    \item A number where the digit 4 has the value 400: \underline{\hspace{4cm}}
\end{enumerate}

Which of your three numbers, when rounded to the nearest thousand, gives 468,000?

\vspace{1cm}

\section*{Self-Assessment}
\begin{tabular}{|l|c|c|c|}
\hline
\textbf{Section} & \textbf{Confident} & \textbf{Need Practice} & \textbf{Need Help} \\
\hline
Words and Figures & $\square$ & $\square$ & $\square$ \\
\hline
Place Value Understanding & $\square$ & $\square$ & $\square$ \\
\hline
Comparing and Ordering & $\square$ & $\square$ & $\square$ \\
\hline
Rounding Numbers & $\square$ & $\square$ & $\square$ \\
\hline
Word Problems & $\square$ & $\square$ & $\square$ \\
\hline
\end{tabular}

\vspace{0.5cm}

\textbf{Parent/Guardian Signature:} \underline{\hspace{5cm}} \textbf{Date:} \underline{\hspace{3cm}}

\section*{Teacher Use Only}
\begin{tabular}{|l|c|l|}
\hline
\textbf{Section A} & \_\_\_/10 & Comments: \\
\hline
\textbf{Section B} & \_\_\_/15 & \\
\hline
\textbf{Section C} & \_\_\_/10 & \\
\hline
\textbf{Section D} & \_\_\_/10 & \\
\hline
\textbf{Section E} & \_\_\_/5 & \\
\hline
\textbf{Extension} & \_\_\_/4 & \\
\hline
\textbf{Total} & \_\_\_/54 & \\
\hline
\end{tabular}

\end{document>